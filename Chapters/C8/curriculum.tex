%\documentclass[11pt]{report}
%\usepackage{siunitx}
%\usepackage{graphicx}
%\usepackage[table,xcdraw]{xcolor}
%\usepackage{csquotes}
%\usepackage{hyperref}
%
%\begin{document}
%\setcounter{chapter}{7}
%\tableofcontents



\newpage
\chapter{Scientific Outputs}
This chapter presents a detailed list of scientific output contributions and communications resulting from this Ph.D. research.
\newpage




\section{Data availability commitment}
Since its beginning, this Ph.D. research has been committed to share designs, precise construction instructions, materials data, tools data, and constructed numerical models. The underlying idea was that, by making these research outputs available to other investigators, the reproducibility and follow-up applications of the developed bioreactor would increase, leading to the adoption and upgrade of the proposed strategy in TE-related areas. Data availability including all mentioned files was achieved through well-established online platforms for scientific data sharing. All the corresponding data links may be found in the respective published manuscript.  


\section{Original research articles}
\begin{enumerate}
\item \small Meneses, João, João C Silva, Sofia R Fernandes, Abhishek Datta, Frederico Castelo Ferreira, Carla Moura, Sandra Amado, Nuno Alves, and Paula Pascoal-Faria. 2020. “A Multimodal Stimulation Cell Culture Bioreactor for Tissue Engineering: A Numerical Modelling Approach”. Polymers 12 (4). DOI: \href{https://doi.org/10.3390/polym12040940}{10.3390/polym12040940}. (SJR: 0.72, IF:5.0, Q1 Chemistry (miscellaneous))
\item \small Meneses, João, Sofia Fernandes, Nuno Alves, Paula Pascoal-Faria, and Pedro Cavaleiro Miranda. 2022. “How to Correctly Estimate the Electric Field in Capacitively Coupled Systems for Tissue Engineering: A Comparative Study”. Scientific Reports 12 (1): 12522. DOI: \href{https://doi.org/10.1038/s41598-022-14834-2}{10.1038/s41598-022-14834-2}. (SJR: 0.97, IF:4.6, Q1 Multidisciplinary)
\item \small João Meneses, Sofia R Fernandes, João C Silva, Frederico Castelo Ferreira, Nuno Alves and Paula Pascoal-Faria. 2023. ''JANUS: an open-source 3D printable perfusion bioreactor and numerical model-based design strategy for tissue engineering''. Front. Bioeng. Biotechnol. 11:1308096. DOI: \href{https://doi.org/10.3389/fbioe.2023.1308096}{10.3389/fbioe.2023.1308096}. (SJR: 0.93, IF:5.7, Q1 Biomedical Engineering)
\item \small João C. Silva \& João Meneses, Fábio F.F. Garrudo, Sofia R. Fernandes, Nuno Alves, Frederico Castelo Ferreira, Paula Pascoal-Faria. ''Direct coupled electrical stimulation towards improved osteogenic differentiation of human mesenchymal stem/stromal cells: a comparative study of different protocols'' (Under Revision at Nature Scientific Reports - SJR: 0.97, IF:4.6, Q1 Multidisciplinary)
\end{enumerate}


\section{Conference Proceedings}
\begin{enumerate}
\item \small Meneses, Joao, Sofia R. Fernandes, Nuno Alves, Paula Pascoal-Faria, and Pedro Cavaleiro Miranda. 2021. “Effects of Scaffold Electrical Properties on Electric Field Delivery in Bioreactors.” Conference Proceedings: Annual International Conference of the IEEE Engineering in Medicine and Biology Society. IEEE Engineering in Medicine and Biology Society. (November): 4147–51. DOI: \href{https://doi.org/10.1109/EMBC46164.2021.9630711}{10.1109/EMBC46164.2021.9630711}
\item \small João Meneses, Carla Moura, Abhisked Datta, Pedro Cavaleiro Miranda, Nuno Alves, Paula Pascoal-Faria. 2021
“The influence of scaffold design in electrical field distribution for tissue engineering.'' Conference Proceedings: The 19th International Conference of Numerical Analysis and Applied Mathematics, Rhodes, Greece.
\item \small Meneses, Joao, Sofia R. Fernandes, Abhishek Datta, Sandra Amado, Nuno Alves, and Paula Pascoal-Faria. 2022. “Numerical Modelling of a Bioreactor Design Targeting Optimal Conditions for Cell Culture.” AIP Conference Proceedings 2425 (1): 220003. DOI: \href{https://doi.org/10.1063/5.0081336}{10.1063/5.0081336}.
\item \small Fernandes, Sofia R., João Meneses, Abhishek Datta, Sandra Amado, Nuno Alves, and Paula Pascoal-Faria. 2022. “Comparison of Electromagnetic Stimulation Fields Generated by Different Experimental Setups: A Biophysical Analysis.” AIP Conference Proceedings 2425 (1): 220004. DOI: \href{https://doi.org/10.1063/5.0081338}{10.1063/5.0081338}.
\end{enumerate}


\section{Oral Presentations}
\begin{enumerate}
\item \small João Meneses, Nuno Alves, Paula Pascoal-Faria. "Electrical Stimulation of Bioscaffolds for Tissue Engineering: a Numerical Analysis" presented at 17\textsuperscript{th} International Conference of Numerical Analysis and Applied Mathematics, held online and presential at Rhodes, Greece, 23-28 September 2019;
\item \small João Meneses, Nuno Alves, Sofia R. Fernandes, Carla Moura, Abhishek Datta, Sandra Amado, P C Miranda, Paula Pascoal-Faria. ''Numerical Modelling of Multi-Coupling Electrodes and Bioreactor Combined System for Electric Stimulation in Tissue Engineering.'' presented at RESIM 2020 / BIODIG 2020 conference held in Centre for Rapid and Sustainable Product Development, Polytechnic Institute of Leiria, 5 June 2020;
\item \small João Meneses, Abhisked Datta, Nuno Alves, Pedro Cavaleiro Miranda and Paula Pascoal-Faria. "Bioreactor Design Challenges and Opportunities: Combining Direct Digital Manufacturing and Numerical Models" presented at ICDDMAP 2021, Session C -Mathematics and Industry, held online, organized by the Centre for Rapid and Sustainable Product Development, Polytechnic Institute of Leiria, and Karnatak University, Dharwad, India. Awarded as Best Presentation in Session C;
\item \small João Meneses, Abhisked Datta, Nuno Alves and Paula Pascoal-Faria. "From Numerical Models to Bioreactor Design in Tissue Engineering" presented at Encontro Nacional da Sociedade Portuguesa de Matemática 2021 (ENSPM 2021), held online, 12-16 July 2021;
\item \small João Meneses, João Silva, Nuno Alves, Tiago Santos, Pedro Cavaleiro Miranda, Paula Pascoal-Faria. “Bioreactor Digital Twin - An essential modelling tool to estimate cellular local environmental conditions in experimental tissue engineering;” presented at International Conference on Computational Bioengineering (ICCB2022), held at Instituto Superior Técnico, Lisbon, Portugal, 11 - 13 April 2022;
\item \small João Meneses, João Silva, Nuno Alves, Tiago Santos, Pedro Cavaleiro Miranda, Paula Pascoal-Faria. “Numerical Modelling Impact in the Design and Operation of Tissue Engineering Systems.” presented at Afternoon Session D - Numerical modelling of the manufacturing of complex systems, RESIM 2022 / BIODIG 2022 conference held in Centre for Rapid and Sustainable Product Development, Polytechnic Institute of Leiria, 3 June 2022;
\item \small João Meneses, Nuno Alves, Pedro Cavaleiro Miranda, Paula Pascoal-Faria. ''Mechanical and EMS in Tissue Engineering. Open Source Bioreactor + Digital Twin Solution.'' Invited lecturer for a weekly research digest at Instituto Superior Técnico, 12 July 2022;
\item \small João Meneses, Sofia R. Fernandes, Nuno Alves, Paula Pascoal-Faria. ''Janus Bioreactor Methodology - A shareable and replicable design for a perfusion bioreactor: decisions on 3D printing supported by a numerical modelling framework.'' presented at Tissue Engineering and Regenerative Medicine International Society (TERMIS) European Chapter Meeting 2023 held in Manchester on the 28th to 31st March 2023. 
\end{enumerate}


\section{Posters}
\begin{enumerate}
\item \small Meneses, João; João C. Silva; Sofia R. Fernandes; Abhishek Datta; Frederico Castelo Ferreira; Carla Moura;
Sandra Amado. "How to improve Tissue Engineering bioreactor solutions to deliver accurate and replicable electromagnetic (EMS) and mechanical stimulation?". Work presented at Ciências Research Day 2020, FCUL, Lisbon, Portugal;
\item \small Meneses, João; João C. Silva; Sofia R. Fernandes; Abhishek Datta; Frederico Castelo Ferreira; Carla Moura;
Sandra Amado."Bioreactor Design: Combining Direct Digital Manufacturing and Numerical Models". Work presented at the Ciência 2021 - Science and Technology in Portugal Summit, 28-30 June 2021, which took place at the Lisbon Congress Centre, Portugal;
\item \small João Meneses, Nuno Alves, Paula Pascoal-Faria, Pedro Cavaleiro Miranda. “Multiscale Numerical Models for Tissue Engineering Applications.” presented at the Jornadas Doutorais, February 2022, which took place at FCUL, Lisbon, Portugal;
\item \small João Meneses, Sofia R. Fernandes, Nuno Alves, Paula Pascoal-Faria. ''Janus Bioreactor - An open-source concept that combines a 3D printable bioreactor with its numerical models, a strategy that may help to improve bioreactor design, operation, control, and overall scientific repeatability.'' presented at Jornadas Doutorais, February 2023, which took place at FCUL, Lisbon, Portugal;
\end{enumerate}


\section{Collaborations}
\begin{enumerate}
\item \small Mateus, J. C., Cdf Lopes, M. Aroso, A. R. Costa, A. Gerós, J. Meneses, P. Faria, et al. 2021. “Bidirectional Flow of Action Potentials in Axons Drives Activity Dynamics in Neuronal Cultures.” Journal of Neural Engineering 18 (6). DOI: \href{https://doi.org/10.1088/1741-2552/ac41db}{10.1088/1741-2552/ac41db};
\item \small Marcelino, Pedro, João Carlos Silva, Carla S. Moura, João Meneses, Rachel Cordeiro, Nuno Alves, Paula Pascoal-Faria, and Frederico Castelo Ferreira. 2023. “A Novel Approach for Design and Manufacturing of Curvature-Featuring Scaffolds for Osteochondral Repair.” Polymers 15 (9). DOI: \href{https://doi.org/10.3390/polym15092129}{10.3390/polym15092129};
\item \small João C. Silva, Pedro Marcelino, João Meneses, Frederico Barbosa, Carla S. Moura,e Ana C. Marques, Joaquim M. S. Cabral, Paula Pascoal-Faria, Nuno Alves, Jorge Morgado, Frederico C. Ferreira, Fábio F. F. Garrudo. ''Synergy between 3D-extruded electroconductive scaffolds and electrical stimulation in enhancing bone regeneration''. (Under Submission at RCS Journal of Materials Chemistry);
\end{enumerate}



%\end{document}